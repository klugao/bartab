\documentclass[a0paper,landscape]{article}
\usepackage[utf8]{inputenc}
\usepackage[portuguese]{babel}
\usepackage[margin=2cm]{geometry}
\usepackage{graphicx}
\usepackage{tikz}
\usepackage{xcolor}
\usepackage{titlesec}
\usepackage{multicol}
% \usepackage{qrcode} % Descomente se tiver o pacote instalado
\usepackage{enumitem}
\usepackage{wrapfig}
\usepackage{ifthen}

% Configuração de cores
\definecolor{sectioncolor}{RGB}{200,0,0} % Vermelho para títulos de seções
\definecolor{titlecolor}{RGB}{0,0,139} % Azul escuro para título principal
\definecolor{textcolor}{RGB}{0,0,0} % Preto para texto

% Configuração de tamanhos de fonte para A0
\newcommand{\hugefont}{\fontsize{72}{86}\selectfont}
\newcommand{\LARGEfont}{\fontsize{48}{58}\selectfont}
\newcommand{\Largefont}{\fontsize{36}{43}\selectfont}
\newcommand{\largefont}{\fontsize{28}{34}\selectfont}
\newcommand{\normalfontsize}{\fontsize{20}{24}\selectfont}
\newcommand{\smallfont}{\fontsize{16}{19}\selectfont}

% Configuração de títulos de seção
\titleformat{\section}
{\color{sectioncolor}\LARGEfont\bfseries}
{}
{0em}
{}

% Remover numeração de seções
\setcounter{secnumdepth}{0}

% Configuração de espaçamento
\setlength{\parskip}{0.5em}
\setlength{\itemsep}{0.3em}

\begin{document}

% ============================================
% CABEÇALHO
% ============================================
\begin{tikzpicture}[remember picture,overlay]
    % Logo da Católica (canto superior esquerdo)
    % Comente a linha abaixo se não tiver a imagem ainda
    \node[anchor=north west] at (current page.north west) {
        \IfFileExists{logo-catolica.png}{%
            \includegraphics[width=8cm,height=auto]{logo-catolica.png}%
        }{%
            \framebox(8cm,3cm){Logo Católica}%
        }
    };
    
    % Título e informações (canto superior direito)
    \node[anchor=north east,align=right] at (current page.north east) {
        \begin{minipage}{25cm}
            {\color{titlecolor}\hugefont\bfseries BarTab}\\[0.5cm]
            {\Largefont Eduardo Vinicios Klug}\\[0.3cm]
            {\largefont eduardo.klug@catolicasc.edu.br}
        \end{minipage}
    };
\end{tikzpicture}

\vspace{2cm}

\begin{center}
{\largefont Graduando do curso de Bacharelado em Engenharia de Software da Católica de SC}
\end{center}

\vspace{1.5cm}

% ============================================
% SEÇÃO 1: INTRODUÇÃO
% ============================================
\section*{Introdução}

\normalfontsize
O cenário de pequenos estabelecimentos comerciais, especialmente bares e botecos, enfrenta desafios significativos na gestão de contas e comandas. O controle manual através de papel, cadernos e calculadoras apresenta riscos de perda de informações, erros no cálculo final e dificuldades na organização do atendimento. Este projeto propõe o desenvolvimento de uma plataforma web de gestão de contas focada em estabelecimentos menores, utilizando arquitetura moderna para escalabilidade e segurança. O sistema inclui registro e gerenciamento de clientes, criação e controle de contas abertas, múltiplos métodos de pagamento (incluindo ``pagar depois'' com controle de dívidas) e relatórios financeiros. A plataforma visa melhorar a eficiência operacional, reduzir erros manuais e fortalecer o controle financeiro dos estabelecimentos.

\vspace{1cm}

% ============================================
% SEÇÃO 2: DESENVOLVIMENTO
% ============================================
\section*{Desenvolvimento}

\begin{multicols}{2}

\textbf{\largefont Front-end:}
\begin{itemize}[leftmargin=*]
    \item React 18.3
    \item Vite
    \item TypeScript
    \item TailwindCSS
    \item Axios
\end{itemize}

\columnbreak

\textbf{\largefont Back-end:}
\begin{itemize}[leftmargin=*]
    \item NestJS 11
    \item TypeORM 0.3
    \item Express
    \item PostgreSQL
\end{itemize}

\end{multicols}

\vspace{0.5cm}

\begin{multicols}{2}

\textbf{\largefont Qualidade:}
\begin{itemize}[leftmargin=*]
    \item Jest (backend)
    \item Vitest (frontend)
    \item SonarCloud (análise estática)
\end{itemize}

\columnbreak

\textbf{\largefont Infraestrutura e DevOps:}
\begin{itemize}[leftmargin=*]
    \item Google Cloud Platform (Cloud Run, Cloud SQL)
    \item GitHub Actions (CI/CD)
    \item Docker
\end{itemize}

\end{multicols}

\vspace{0.5cm}

% Espaço para logos das tecnologias
\begin{center}
% Inclua aqui os logos das tecnologias
% \includegraphics[height=2cm]{logo-react.png}
% \includegraphics[height=2cm]{logo-nestjs.png}
% \includegraphics[height=2cm]{logo-postgresql.png}
% \includegraphics[height=2cm]{logo-gcp.png}
% etc...
\end{center}

\vspace{1cm}

% ============================================
% SEÇÃO 3: RESULTADO
% ============================================
\section*{Resultado}

\normalfontsize
O sistema foi implantado com sucesso no Google Cloud Platform utilizando serviços como Cloud Run para escalabilidade automática e Cloud SQL para gerenciamento do banco de dados PostgreSQL. A qualidade do software foi garantida através de testes automatizados (Jest/Vitest) com 126 testes implementados (51 backend + 75 frontend) e análise estática contínua via SonarCloud, integrada em pipeline de CI/CD. A plataforma é segura, com autenticação OAuth via Google e controle de acesso baseado em roles (RBAC), capaz de gerenciar todo o ciclo de vida de contas e pagamentos com monitoramento de métricas de negócio em tempo real.

\vspace{0.5cm}

\begin{center}
% Screenshot da interface
\IfFileExists{screenshot-interface.png}{%
    \includegraphics[width=0.8\textwidth,height=0.4\textheight,keepaspectratio]{screenshot-interface.png}%
}{%
    \framebox(0.8\textwidth,0.4\textheight){Screenshot da Interface}%
}
\end{center}

\vspace{0.3cm}

{\smallfont\textit{Painel de gestão de contas}}

\vspace{0.3cm}

{\smallfont O módulo de gestão permite controle autônomo de contas, clientes e pagamentos. A interface reflete a arquitetura do banco de dados, garantindo que cada conta esteja corretamente vinculada a um cliente e validada por middleware de segurança para prevenir acesso não autorizado aos recursos.}

\vspace{1cm}

% ============================================
% SEÇÃO 4: QR CODE
% ============================================
\begin{center}
\section*{QR code para acessar a aplicação}

\vspace{0.5cm}

% QR Code - use uma imagem externa ou descomente a linha abaixo se tiver o pacote qrcode
\IfFileExists{qrcode.png}{%
    \includegraphics[width=8cm,height=8cm,keepaspectratio]{qrcode.png}%
}{%
    \framebox(8cm,8cm){QR Code}%
}
% \qrcode[height=8cm]{https://seu-dominio.com} % Descomente se tiver o pacote qrcode instalado

\vspace{0.3cm}

{\smallfont https://seu-dominio.com} % Substitua pela URL real
\end{center}

\vspace{1cm}

% ============================================
% SEÇÃO 5: PRINCIPAL CASO DE USO
% ============================================
\section*{Principal caso de uso}

\begin{center}
% Diagrama UML de casos de uso
\IfFileExists{diagrama-casos-uso.png}{%
    \includegraphics[width=0.7\textwidth,height=0.35\textheight,keepaspectratio]{diagrama-casos-uso.png}%
}{%
    \framebox(0.7\textwidth,0.35\textheight){Diagrama de Casos de Uso}%
}
\end{center}

\vspace{0.3cm}

{\smallfont\textit{BarTab - Atendente/Gerente}}

\vspace{1cm}

% ============================================
% SEÇÃO 6: ARQUITETURA
% ============================================
\section*{Arquitetura}

\begin{center}
% Diagrama de arquitetura
\IfFileExists{diagrama-arquitetura.png}{%
    \includegraphics[width=0.9\textwidth,height=0.4\textheight,keepaspectratio]{diagrama-arquitetura.png}%
}{%
    \framebox(0.9\textwidth,0.4\textheight){Diagrama de Arquitetura}%
}
\end{center}

\vspace{0.3cm}

{\smallfont\textit{CI/CD Pipeline (GitHub Actions) e Google Cloud Platform (Infraestrutura)}}

\vspace{1cm}

% ============================================
% SEÇÃO 7: CONCLUSÃO
% ============================================
\section*{Conclusão}

\normalfontsize
O desenvolvimento da plataforma BarTab validou a aplicação prática de conceitos avançados de Engenharia de Software em uma solução real. O objetivo principal de criar um sistema escalável de gestão de contas foi alcançado através de uma arquitetura em camadas e segura. As principais lições aprendidas incluem a complexidade de orquestrar ambientes em nuvem (GCP) e a importância crítica da automação (CI/CD) para manter a qualidade do software. Conclui-se que a adoção de padrões de projeto (como Repository Pattern) e estratégias rigorosas de testes são fundamentais para garantir a robustez e manutenibilidade de sistemas modernos multi-tenant.

\vspace{1cm}

% ============================================
% SEÇÃO 8: REFERÊNCIAS
% ============================================
\section*{Referências}

{\smallfont
\begin{enumerate}[leftmargin=*]
    \item NESTJS. NestJS - A progressive Node.js framework. Disponível em: https://nestjs.com/. Acesso em: nov. 2025.
    
    \item REACT. React: The library for web and native user interfaces. Disponível em: https://react.dev/. Acesso em: nov. 2025.
    
    \item GOOGLE CLOUD PLATFORM. Cloud Run Documentation. Disponível em: https://cloud.google.com/run/docs. Acesso em: nov. 2025.
    
    \item POSTGRESQL. PostgreSQL: The World's Most Advanced Open Source Relational Database. Disponível em: https://www.postgresql.org/. Acesso em: nov. 2025.
    
    \item TYPEORM. TypeORM - Data Mapper, Active Record patterns. Disponível em: https://typeorm.io/. Acesso em: nov. 2025.
    
    \item JEST. Jest - Delightful JavaScript Testing. Disponível em: https://jestjs.io/. Acesso em: nov. 2025.
    
    \item VITEST. Vitest - Next Generation Testing Framework. Disponível em: https://vitest.dev/. Acesso em: nov. 2025.
    
    \item SONARCLOUD. SonarCloud - Clean Code. Disponível em: https://www.sonarsource.com/products/sonarcloud/. Acesso em: nov. 2025.
\end{enumerate}
}

\end{document}

